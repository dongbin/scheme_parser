\documentclass{article}
\XeTeXlinebreaklocale "zh"
\XeTeXlinebreakskip = 0pt plus 1pt minus 0.1pt

\usepackage{fontspec}
\usepackage{color}
\usepackage{xunicode}
\usepackage{xltxtra}
\usepackage{listings}

\linespread{1.36}
\setromanfont{AR PL ShanHeiSun Uni}

\lstset{
framexleftmargin=5mm,frame=single,
keywordstyle=\color{blue!70}}

\title{编写 Scheme 解释器 Step by Step}
\author{董彬}
\date{\today}

\begin{document}
\maketitle

\section{引言}
亲手实现一个解释器能够让你深刻理解解释器的工作原理,对于理解编程本质很有帮助。本文将简单介绍解释器的实现原理,并且教你使用 Ruby
语言和 TreeTop 框架简单快速地实现一个 Scheme 解释器。

\section{Scheme 语言}
Scheme 是 Lisp 的一个方言。Lisp 是最古老的现存编程语言之一,具有三十多
年的历史,随着函数式编程的兴起,LISP社区又在展现勃勃生机。

为什么笔者要实现一个 Scheme 解释器呢?这要从“Structure and
Interpretation of Computer Programs”(简称 SICP)\footnote{中文版《计
  算机程序的构造和解释》,机械工业出版社出版}这本书谈起。这本书是MIT程
序设计的本科生教科书,让我对对程序设计方法学有了深刻理解。SICP 之所以选择 Lisp所谓教学语言,是因为Lisp具有最简单的语法结
构和高度的灵活性。

此书第4章讲解了如何实现一个Lisp解释器,文中的实现语言也是Lisp。整章解
释器实现的代码非常详尽,但是篇幅太长。读者很难跟随此书亲自编写解释器。
笔者是一个Ruby语言爱好者,所以决定采用 Ruby 语言并且借用语法分析工具 TreeTop
,用最少的代码来快速实现一个Lisp解释器。以便让读者能够跟随笔者的脚步来
动手实践,实现一个Lisp解释器。
\subsection{ Scheme表达式 }
Scheme表达式是一对括号括起来的一系列元素,最左面的元素称为运算符,其他
元素称为运算对象。表达式的求值结果就是将运算符所表达的过程应用于运算对
象。例如:
\begin{verbatim}
(+ 1 2)
-> 3
\end{verbatim}
表达式也可以嵌套,也就是说运算符或者运算对象也可以是表达式
\begin{verbatim}
(+ 1 (+ 1 2))
-> 4
\end{verbatim}
\subsection{变量和过程的定义}
在 Scheme 中, 通过 define 关键字来定义变量:
\begin{verbatim}
(define pi 3.14)
\end{verbatim}
然后通过名字引用变量
\begin{verbatim}
pi
-> 3.14
\end{verbatim}
过程的定义跟变量类似:
\begin{verbatim}
(define (square x) (* x x))
\end{verbatim}
定义之后就可以使用了:
\begin{verbatim}
(square 2)
-> 4
\end{verbatim}
Lisp中的符号既可以绑定数值,还可以绑定过程。
\section{Treetop介绍}

\subsection{Ruby的语法树解析器}

编程语言的解释和自然语言类似,由两部分组成:语法和语义。语法代表语句结
构,好比自然语言的主谓宾。语义是语句的含义。解释器需要首先对输入的字符
序列构造一棵抽象语法树,然后执行每个节点对应的语义。使用 Ruby 生成语
法树有多种方案,笔者尝试了以下几种:
\begin{enumerate}
\item Racc:Yacc的Ruby实现,包含在 Ruby 安装包中,兼容 YACC 语法,可以
  生成语法解析的Ruby实现。缺点是 YACC 语法太复杂,初学者不易掌握。
\item Recursive Descent Parser ,不生成 Parser 代码,而是在框架基础上
  手工编写 Parser,用Ruby语言来描述语法。优点是不需要生成代码,便于维护。缺点 是受到 Ruby
  语言的限制,语法描述不够简洁。
\item Treetop:开源的语法树生成器,通过 gem 安装。Treetop的优点是:
  \begin{itemize}
  \item 使用 PEG (Parsing Expression Grammar)来描述语法,简单易懂。
  \item 通过节点绑定Ruby类的方式,可以使用面向对象的方式来组织和封装解释器代码。
  \end{itemize}

\end{enumerate}
经过比较,笔者选用 Treetop 作为 Parser 生成器。
\subsection{使用Treetop进行语法识别}

Treetop的语法结构如下:
\begin{lstlisting}
#my_grammar.treetop
grammar MyGrammar
  rule number
    ... ...
  end
  rule string
    ... ...
  end
end
\end{lstlisting}

在 my\_grammar.treetop 中,有两个关键字:
\begin{itemize}
\item grammar : 定义一个新语法
\item rule : 定义一个语法规则,后面跟着一个可以被其他规则所引用的名字。
\end{itemize}
每个 rule 关联一个名字和一个解析表达式(Parsing Expression,以下简称PE)。PE是一种正则表达式的变种,其最重要的特性是可以在表达式中通过名字引用其
他解析表达式。根据可否引用其他表达式,PE可以分为两种:
Terminal Symbols 和 Nonterminal Symbols

\subsubsection{Terminal Symbols}
Terminal Symbols 指的是原子表达式,包括字符串和字符:
\begin{itemize}
\item 字符串由单引号或者双引号括起来,比如 'hello',“Hello”
\item 字符跟正则表达式中的字符一样,比如 [0-9],[a-z]
\end{itemize}

\subsubsection{出现次数}
跟正则表达式一样,PE可以描述出现次数。
\begin{itemize}
\item 'foo'* 零次或多次
\item 'foo'+ 一次或多次
\item 'foo'? 一次或零次
\end{itemize}

\subsubsection{Nonterminal Symbols}
Nonterminal Symbols 指的是表达式中引用其他 rule,引用的效果等价于C语言
中的宏。例如:

\begin{lstlisting}
rule hello
  "hello " name
end

rule name
  "dongbin"
end
\end{lstlisting}

等价于用name中描述的模式替换hello中对name的引用:

\begin{lstlisting}
rule hello
  "hello dongbin"
end
\end{lstlisting}

\subsubsection{元素选择}
解析器可以从一系列元素中顺序匹配,直到找到匹配的元素为止。比如
\begin{lstlisting}
  "foobar" / "foo" / "bar"
\end{lstlisting}
会匹配 "foobar" ,"foo" 或者 “bar”。注意解析器按照从左到右的顺序进行匹
配,元素的顺序会影响匹配结果。

\section{编写Scheme解释器}

\subsection{解释器的结构}

解释器的执行原理是:
\begin{itemize}
\item 根据语法定义,把输入的文本转换成一棵语法树。
\item 根据语义定义,对根节点求值,这个求值过程会递归地对子节点求值,直到返回数值。
\end{itemize}

对于每一个语法节点,要分别定义语法和语义:
\begin{description}
  \item[语法定义] 描述语法节点的结构。如果使用 Treetop,就是 \em{rule} 代码块中的PE。
  \item[语义定义] 描述语法节点的求值过程。在本文中,每个语法节点都会有
    一个名为\em{eval}的Ruby函数来求值,该方法接受一个Hash类型参数,作为求值的环境。。
\end{description}

环境是什么? 在这里引用一下 SICP 中环境的定义:\em{我们可以将值与符号
  关联,而后又能提取出这些值,这意味着解释器必须维护某种存储能力,以便
  保持有关的“名字-值”对偶的轨迹。这种存储被称为环境。}\footnote{《计算
    机程序的构造和解释》,P5-6}
\subsection{语义解析}
语义解析就是对语法节点进行求值的过程。如何使用Treetop对语法节点进行求值呢?举个例子,我们要求值:
\begin{verbatim}
(+ 1 (+ 3 2))
\end{verbatim}
我们的Parser就会生成类似  一棵语法树。树中每一个节点都是一个
Treetop::Runtime::SyntaxNode 实例。Treetop 允许我们在每一个节点对象上
自定义方法,这样节点的操作就可以被封装在SyntaxNode的定义中,我们就可以
按照面向对象的方式组织解释器的结构。

SyntaxNode 类有一些内建的方法,本文只用到了 text\_value 方法。此方法返
回语法节点的被匹配到的文本形式。例如
\begin{verbatim}
  rule number
    [1-9] [0-9]*
  end
\end{verbatim}
匹配到 \em{10} 这个文本,那么text\_value返回字符串“10”。


Treetop提供了两种方式来自定义 SyntaxNode :
\begin{enumerate}
\item 内联方法定义。在 treetop文件中,某个rule对应的方法定义可以写在
  rule的定义内部:
\begin{lstlisting}
  rule number
    [1-9] [0-9]* {
      def eval(env={})
        text_value.to_f
      end
    }
  end
\end{lstlisting}
这个 number rule 生成的 SyntaxNode 实例就具有 eval 方法。
\item 子类定义。如果某个rule所对应的Ruby代码过多,放在grammer文件中太
  臃肿,我们可以把Ruby代码写在另一个 .rb 文件里面,然后用尖括号来引用
  类名。
\begin{lstlisting}
  rule number
    [1-9] [0-9]* <Number>
  end

  # nodes.rb
class NumberNode < Treetop::Runtime::SyntaxNode
  def eval(env={})
    text_value.to_f
  end
end
\end{lstlisting}

\end{enumerate}
背景知识介绍到这里,下面进入编写Scheme解释器的实战阶段。Let's go!

\subsection{语法定义}
下面我们开始以自顶向下地定义语法。Treetop规定,\em{grammer} 中定义的第
一个 \em{rule} 是 语法树的根节点。它描述了输入文本的总体语法结构。对于 Scheme 方言,根节点是由一个或者多个表
达式(expression) 组成:
\begin{verbatim}
grammar Scheme

  rule program
    expression expressions <Program>
  end

  rule expressions
    expression*
  end

end
\end{verbatim}

Program 类描述了一个完整 Lisp 程序求值的过程,它的定义如下:
\begin{verbatim}
class Program < Treetop::Runtime::SyntaxNode
  def eval(env={})
    exprs.inject(nil){ |last_env, exp|
      exp.eval(env)
    }
  end

  def exprs
    [expression] + expressions.exprs
  end

end
\end{verbatim}
\subsubsection{expression 定义}
Scheme的表达式分为以下几种:
\begin{description}
\item[数值]数字,字符串等可以直接求值的元素,分别用 \em{number}和
  \em{string}来表示。
\item[符号]需要从环境中查找符号对应的数值,用 \em{symbol}表示。
\item[原子过程]可以直接求值的运算,包括 +, -,*和/,用 \em{primary}表
  示

\item[复合过程]不能直接求值,需要解释器将过程定义应用于实际参数才能求
  值,用
  \em{function}表示。
\item[变量定义] 定义变量的语法,用 \em{defination}表示。
\item[过程定义] 定义过程的语法,用 \em{define\_function}表示。
\end{description}
用PE描述 expression 的结构如下:

\begin{verbatim}
  rule expression
    # 给 expression 中可能出现的元素起个别名,便于在 eval 函数中引用。
    space e:(atom / primary/ defination / define_function / function) space {
      def eval(env = {})
        e.eval(env)
      end
    }
  end
  rule space
    [ \n]*
  end
  rule atom
    string / number / symbol
  end

\end{verbatim}
为了定义 expression rule,我们又新引入了两个 rule:
\begin{itemize}
\item 用 atom 来代表number,string或者symbol。使用 Nonterminal
Symbols来组织和分类rule能够提高grammer的可读性和可维护性。
\item 用 space 来代表空白。语法定义中常常会出现这种字符串常量,定义
  rule来描述常量能够提高灵活性。
\end{itemize}

\subsubsection{atom 定义}
下面定义 string和number, symbol的定义我们稍后再讲。
\begin{verbatim}
  rule string
    '"' (!'"' . / '\"')* '"' {
      def eval(env={})
        text_value[1..-2]
      end
    }
  end

  rule number
    [1-9] [0-9]* {
      def eval(env={})
        text_value.to_f
      end
    }
  end

\end{verbatim}

\subsubsection{原子过程定义}
对于原子过程,我们只定义加减乘除四个过程。
\begin{verbatim}
  rule primary
    plus / minus / multiply / divide
  end

  rule divide
    lparen '/' dividend:expression divisor:expression  rparen {
      def eval(env={})
        dividend.eval(env) / divisor.eval(env)
      end
    }
  end

  rule multiply
    lparen '*' expressions rparen {
      def eval(env={})
        expressions.exprs_value(env).inject{ |a, b| a * b}
      end
    }
  end

  rule minus
    lparen '-' minuend:expression substractor:expression rparen {
      def eval(env={})
        minuend.eval(env) - substractor.eval(env)
      end
    }
  end

  rule plus
    lparen '+' expressions  rparen {
      def eval(env={})
        expressions.exprs_value(env).inject{ |a, b| a + b}
      end
    }
  end

  rule lparen
    space '(' space
  end

  rule rparen
    space ')' space
  end

\end{verbatim}

\subsubsection{变量的定义和求值}
变量的定义过程就是向环境中插入一个\em{key-value}对,其中key是变量
名,value是字符串,数值,或者一个过程。
\begin{verbatim}
  rule defination
    lparen define symbol space expression rparen {
      def eval(env)
        env[symbol.name] = expression.eval(env)
      end
    }
  end

  rule define
    space 'define' space
  end


\end{verbatim}
变量求值就是把值从环境中取出来。取值的过程就是symbol求值的过程,下面定
义symbol:
\begin{verbatim}
  rule symbol
    space [a-z] ([a-z] / [0-9] / [_])* {
      def eval(env)
        env[name]
      end
      def name
        text_value.strip
      end
    }
  end

\end{verbatim}

\subsubsection{复合过程}
最复杂的部分就是复合过程的定义和求值。复合过程的求值
就是一个 eval 和 apply互相调用的过程。符合过程的定义跟变量的定义类似,
只不过变量的值是一个 CompoundProcedure 对象的实例。创建
CompoundProcedure 对象需要三个参数:环境,形参,过程体。

\begin{verbatim}
  rule define_function
    lparen define lparen operator:symbol args:symbol+ rparen  body:expression rparen {
      def eval(env)
        env[operator.name] = CompoundProcedure.new(env, args, body)
      end
    }
  end

\end{verbatim}
CompoundProcedure 的定义如下:
\begin{verbatim}
class CompoundProcedure < Treetop::Runtime::SyntaxNode

  def initialize(env, args, body)
    @env = env
    @args = args.elements.map{ |arg| arg.name}
    @body = body
  end

  def apply(*values)
    raise SchemeSyntaxError.new("Parameter doesn't match") unless @args.size == values.size
    @body.eval(extend_env(values))
  end

  private

  def extend_env(values)
    @env.merge([@args, values].transpose.
               map{ |key, value| { key => value}}.
               inject{|sum, value| sum.merge(value)})
  end
end

\end{verbatim}

复合过程的调用:
\begin{verbatim}
  rule function
    lparen operator:expression operands:expressions rparen {
      def eval(env)
        operator.eval(env).
          apply(*(operands.elements.map{ |value| value.eval }))
      end
    }
  end
\end{verbatim}
\section{测试}

我们还没有编写命令行环境来运行这个简单的解释器,不过可以编写测试代码来
测试解释器。

\end{document}
